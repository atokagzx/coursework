\documentclass[12pt,a4paper]{article}
\usepackage[utf8]{inputenc}
\usepackage[russian]{babel}
\usepackage[T2A]{fontenc}
\usepackage{geometry}
\geometry{a4paper, total={170mm,257mm}, left=20mm, top=20mm}
\usepackage{graphicx}
\usepackage{hyperref}

\title{Разработка Telegram-бота для работы с видеофайлами}
\author{Ваше Имя}
\date{\today}

\begin{document}

\maketitle
\newpage

\tableofcontents
\newpage

\section{Пользовательское описание}
Программа представляет собой Telegram-бота, который позволяет пользователю взаимодействовать с видеофайлами различных форматов, таких как MKV, MP4, AVI, WMV, MOV и др. В качестве дополнения к видео пользователь может добавить текстовое описание, поддерживающее символы в формате Unicode. Кроме того, от пользователя требуется указать его имя или никнейм для идентификации.

\subsection{Описание при работе с ботом}
Для начала работы с ботом необходимо найти его в мессенджере Telegram по имени \texttt{@coursework\_video\_bot} и открыть чат. При первом открытии чата с ботом отображается краткая информация о его возможностях. Нажатие на кнопку "Начать" приведет к появлению сообщения с кратким руководством по функционалу бота.

Если пользователь взаимодействует с ботом впервые, он получит предложение представиться. Для этого следует отправить сообщение с командой \texttt{</setname [имя]>}. В случае необходимости изменения имени можно повторно использовать команду \texttt{/setname}.

После регистрации имени пользователь может отправить видеоролик. Если видео проходит предварительную модерацию, бот отвечает сообщением "Видео принято". В противном случае будет выведено сообщение об ошибке. Успешная загрузка видео дает возможность добавить к файлу текстовое описание, отправив текстовое сообщение в ответ на ранее загруженное видео.

\textbf{Установленные лимиты:}
\begin{itemize}
    \item Размер файла – не более 20 кбайт.
    \item Длительность видео – не более 30 секунд.
\end{itemize}

\end{document}
